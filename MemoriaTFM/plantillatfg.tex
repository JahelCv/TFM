%%%%%%%%%%%%%%%%%%%%%%%%%%%%%%%%%%%%%%%%%%%%%%%%%%%%%%%%%%%%%%%%%%%%%%%%%%%%%%%
%                       CARREGA DE LA CLASSE DE DOCUMENT                      %
%                                                                             %
% Les opcions admissibles son:                                                %
%      12pt / 11pt            (cos dels tipus de lletra; no feu servir 10pt)  %
%                                                                             %
% catalan/spanish/english     (llengua principal del treball)                 %
%                                                                             % 
% french/italian/german...    (si necessiteu fer servir alguna altra llengua) %
%                                                                             %
% listoffigures               (El document inclou un Index de figures)        %
% listoftables                (El document inclou un Index de taules)         %
% listofquadres               (El document inclou un Index de quadres)        %
% listofalgorithms            (El document inclou un Index d'algorismes)      %
%                                                                             %
%%%%%%%%%%%%%%%%%%%%%%%%%%%%%%%%%%%%%%%%%%%%%%%%%%%%%%%%%%%%%%%%%%%%%%%%%%%%%%%

\documentclass[12pt,spanish,listoffigures,listoftables]{tfgetsinf}

%%%%%%%%%%%%%%%%%%%%%%%%%%%%%%%%%%%%%%%%%%%%%%%%%%%%%%%%%%%%%%%%%%%%%%%%%%%%%%%
%                     CODIFICACIO DEL FITXER FONT                             %
%                                                                             %
%    windows fa servir normalment 'ansinew'                                   %
%    amb linux es possible que siga 'latin1' o 'latin9'                       %
%    Pero el mes recomanable es fer servir utf8 (unicode 8)                   %
%                                          (si el vostre editor ho permet)    % 
%%%%%%%%%%%%%%%%%%%%%%%%%%%%%%%%%%%%%%%%%%%%%%%%%%%%%%%%%%%%%%%%%%%%%%%%%%%%%%%

\usepackage[utf8]{inputenc} 

%%%%%%%%%%%%%%%%%%%%%%%%%%%%%%%%%%%%%%%%%%%%%%%%%%%%%%%%%%%%%%%%%%%%%%%%%%%%%%%
%                        ALTRES PAQUETS I DEFINICIONS                         %
%                                                                             %
% Carregueu aci els paquets que necessiteu i declareu les comandes i entorns  %
%                                          (aquesta seccio pot ser buida)     %
%%%%%%%%%%%%%%%%%%%%%%%%%%%%%%%%%%%%%%%%%%%%%%%%%%%%%%%%%%%%%%%%%%%%%%%%%%%%%%%



%%%%%%%%%%%%%%%%%%%%%%%%%%%%%%%%%%%%%%%%%%%%%%%%%%%%%%%%%%%%%%%%%%%%%%%%%%%%%%%
%                        DADES DEL TREBALL                                    %
%                                                                             %
% titol, alumne, tutor i curs academic                                        %
%%%%%%%%%%%%%%%%%%%%%%%%%%%%%%%%%%%%%%%%%%%%%%%%%%%%%%%%%%%%%%%%%%%%%%%%%%%%%%%

\title{Interconexión del robot NAO a servicios de simulación en la nube}
\author{Carmona Vila, Jahel}
\tutor{Blanes Noguera, Juan Francisco}
\curs{2018-2019}

%%%%%%%%%%%%%%%%%%%%%%%%%%%%%%%%%%%%%%%%%%%%%%%%%%%%%%%%%%%%%%%%%%%%%%%%%%%%%%%
%                     PARAULES CLAU/PALABRAS CLAVE/KEY WORDS                  %
%                                                                             %
% Independentment de la llengua del treball, s'hi han d'incloure              %
% les paraules clau i el resum en els tres idiomes                            %
%%%%%%%%%%%%%%%%%%%%%%%%%%%%%%%%%%%%%%%%%%%%%%%%%%%%%%%%%%%%%%%%%%%%%%%%%%%%%%%

\keywords{NAO, simulador, diabetes, HTTP, MQTT} % Paraules clau 
         {NAO, simulador, diabetes, HTTP, MQTT} % Palabras clave
         {NAO, simulator, diabetes, HTTP, MQTT}        % Key words

%%%%%%%%%%%%%%%%%%%%%%%%%%%%%%%%%%%%%%%%%%%%%%%%%%%%%%%%%%%%%%%%%%%%%%%%%%%%%%%
%                              INICI DEL DOCUMENT                             %
%%%%%%%%%%%%%%%%%%%%%%%%%%%%%%%%%%%%%%%%%%%%%%%%%%%%%%%%%%%%%%%%%%%%%%%%%%%%%%%

\begin{document}

\begin{abstract}
En aquest projecte es desenvolupen serveis en el núvol per tal de desacoblar de dins del robot NAO el codi que executa el simulador de diabetes que duu implementat. Es suprimeixen les connexions TCP per protocols de capes superiors en la pila TCP/IP com son HTTP i MQTT que permeten una comunicació més flexible i amb més opcions. A més, es modernitza el codi de manera que sigue més senzilla tant la seva estructura com la seva compilació i execució.
\end{abstract}
\begin{abstract}[spanish]
En este proyecto se desarrollan servicios en la nube para desacoplar de dentro del robot NAO el código que ejecuta el simulador de diabetes que lleva implementado. Se suprimen las conexiones TCP por protocolos de capas superiores en la pila TCP/IP como son HTTP y MQTT que permiten una comunicación más flexible y con más opciones. Además, se moderniza el código para que sea más sencilla tanto su estructura como su compilación y ejecución.
\end{abstract}
\begin{abstract}[english]
In this project are developed cloud services to disengage the code inside the NAO Robot that starts up the diabetes simulator. TCP connectios are suppressed by higher layer protocols in the TCP/IP stack as HTTP and MQTT are which allow a more flexible communication and with more options. Besides the code is modernized so that this can be simplier in terms of structure, compilation and execution.
\end{abstract}

%%%%%%%%%%%%%%%%%%%%%%%%%%%%%%%%%%%%%%%%%%%%%%%%%%%%%%%%%%%%%%%%%%%%%%%%%%%%%%%
%                              CONTINGUT DEL TREBALL                          %
%%%%%%%%%%%%%%%%%%%%%%%%%%%%%%%%%%%%%%%%%%%%%%%%%%%%%%%%%%%%%%%%%%%%%%%%%%%%%%%

\mainmatter

%%%%%%%%%%%%%%%%%%%%%%%%%%%%%%%%%%%%%%%%%%%%%%%%%%%%%%%%%%%%%%%%%%%%%%%%%%%%%%%
%                                  INTRODUCCIO                                %
%%%%%%%%%%%%%%%%%%%%%%%%%%%%%%%%%%%%%%%%%%%%%%%%%%%%%%%%%%%%%%%%%%%%%%%%%%%%%%%

\chapter{Introducción}

Idees de estructura:
- Explicació del NAO i tipos de llenguatge
- Explicació del projecte anterior i arquitectura (dibuixets de TCP)
- Explicació de les limitacions de còmput de NAO i de limitacions de TCP
- Estructura del meu projecte. MQTT entrar en detall del QoS. (Part Java, Part NAO i part GCloud).
- Part Java: Mínim canvi realment. Explicar Paho i que es un client Rest.
- Part NAO: Que he reconvertit el codi a Python perquè el NAO te molt mal el C++. 
- Part GCloud: Aquí explayar-se.
- Millores: Prescindir de GCloud i montarse una raspi amb un DNS, afegir més funcions al NAO, mes precisio del simulador aprofitant gcloud, manejar varios NAOs amb una sola app java i un sol gcloud...

\section{Motivación}

????? ????????????? ????????????? ????????????? ????????????? ????????????? 

\section{Objectius}

????? ????????????? ????????????? ????????????? ????????????? ?????????????

\section{Estructura de la mem\`oria}

????? ????????????? ????????????? ????????????? ????????????? ????????????? 

%\section{Notes bibliografiques} %%%%% Opcional

%????? ????????????? ????????????? ????????????? ????????????? ?????????????

%%%%%%%%%%%%%%%%%%%%%%%%%%%%%%%%%%%%%%%%%%%%%%%%%%%%%%%%%%%%%%%%%%%%%%%%%%%%%%%
%                         CAPITOLS (tants com calga)                          %
%%%%%%%%%%%%%%%%%%%%%%%%%%%%%%%%%%%%%%%%%%%%%%%%%%%%%%%%%%%%%%%%%%%%%%%%%%%%%%%

\chapter{??? ???? ??????}

????? ????????????? ????????????? ????????????? ????????????? ?????????????

\section{?? ???? ???? ? ?? ??}

????? ????????????? ????????????? ????????????? ????????????? ?????????????

\chapter{??? ???? ??????}

????? ????????????? ????????????? ????????????? ????????????? ????????????? 

\section{?? ???? ???? ? ?? ??}

????? ????????????? ????????????? ????????????? ????????????? ?????????????

%%%%%%%%%%%%%%%%%%%%%%%%%%%%%%%%%%%%%%%%%%%%%%%%%%%%%%%%%%%%%%%%%%%%%%%%%%%%%%%
%                                 CONCLUSIONS                                 %
%%%%%%%%%%%%%%%%%%%%%%%%%%%%%%%%%%%%%%%%%%%%%%%%%%%%%%%%%%%%%%%%%%%%%%%%%%%%%%%

\chapter{Conclusions}

????? ????????????? ????????????? ????????????? ????????????? ????????????? 

%%%%%%%%%%%%%%%%%%%%%%%%%%%%%%%%%%%%%%%%%%%%%%%%%%%%%%%%%%%%%%%%%%%%%%%%%%%%%%%
%                                BIBLIOGRAFIA                                 %
%%%%%%%%%%%%%%%%%%%%%%%%%%%%%%%%%%%%%%%%%%%%%%%%%%%%%%%%%%%%%%%%%%%%%%%%%%%%%%%

\begin{thebibliography}{10}

%%%%%%%%%%%%%%%%%%%%%%%%%%%%%%%%%%%%%%%%%%%%%%%%%%%%%%%%%%%%%%%%%%%%%%%%%%%%%%%
% MODEL D'ARTICLE                                                             %
%%%%%%%%%%%%%%%%%%%%%%%%%%%%%%%%%%%%%%%%%%%%%%%%%%%%%%%%%%%%%%%%%%%%%%%%%%%%%%%
\bibitem{light}
   Jennifer~S. Light.
   \newblock When computers were women.
   \newblock \textit{Technology and Culture}, 40:3:455--483, juliol, 1999.

%%%%%%%%%%%%%%%%%%%%%%%%%%%%%%%%%%%%%%%%%%%%%%%%%%%%%%%%%%%%%%%%%%%%%%%%%%%%%%%
% MODEL DE LLIBRE                                                             %
%%%%%%%%%%%%%%%%%%%%%%%%%%%%%%%%%%%%%%%%%%%%%%%%%%%%%%%%%%%%%%%%%%%%%%%%%%%%%%%
\bibitem{ifrah}
   Georges Ifrah.
   \newblock \textit{Historia universal de las cifras}.
   \newblock Espasa Calpe, S.A., Madrid, sisena edició, 2008.

%%%%%%%%%%%%%%%%%%%%%%%%%%%%%%%%%%%%%%%%%%%%%%%%%%%%%%%%%%%%%%%%%%%%%%%%%%%%%%%
% MODEL D'URL                                                                 %
%%%%%%%%%%%%%%%%%%%%%%%%%%%%%%%%%%%%%%%%%%%%%%%%%%%%%%%%%%%%%%%%%%%%%%%%%%%%%%%
\bibitem{WAR}
   Comunicat de premsa del Departament de la Guerra, 
   emés el 16 de febrer de 1946. 
   \newblock Consultat a 
   \url{http://americanhistory.si.edu/comphist/pr1.pdf}.

\end{thebibliography}
\cleardoublepage

%%%%%%%%%%%%%%%%%%%%%%%%%%%%%%%%%%%%%%%%%%%%%%%%%%%%%%%%%%%%%%%%%%%%%%%%%%%%%%%
%                           APÈNDIXS  (Si n'hi ha!)                           %
%%%%%%%%%%%%%%%%%%%%%%%%%%%%%%%%%%%%%%%%%%%%%%%%%%%%%%%%%%%%%%%%%%%%%%%%%%%%%%%

\APPENDIX

%%%%%%%%%%%%%%%%%%%%%%%%%%%%%%%%%%%%%%%%%%%%%%%%%%%%%%%%%%%%%%%%%%%%%%%%%%%%%%%
%                         LA CONFIGURACIO DEL SISTEMA                         %
%%%%%%%%%%%%%%%%%%%%%%%%%%%%%%%%%%%%%%%%%%%%%%%%%%%%%%%%%%%%%%%%%%%%%%%%%%%%%%%

\chapter{Configuració del sistema}

????? ????????????? ????????????? ????????????? ????????????? ?????????????

\section{Fase d'inicialització}

????? ????????????? ????????????? ????????????? ????????????? ?????????????

\section{Identificació de dispositius}

????? ????????????? ????????????? ????????????? ????????????? ?????????????

%%%%%%%%%%%%%%%%%%%%%%%%%%%%%%%%%%%%%%%%%%%%%%%%%%%%%%%%%%%%%%%%%%%%%%%%%%%%%%%
%                               ALTRES  APÈNDIXS                              %
%%%%%%%%%%%%%%%%%%%%%%%%%%%%%%%%%%%%%%%%%%%%%%%%%%%%%%%%%%%%%%%%%%%%%%%%%%%%%%%


\chapter{??? ???????????? ????}

????? ????????????? ????????????? ????????????? ????????????? ????????????? 



%%%%%%%%%%%%%%%%%%%%%%%%%%%%%%%%%%%%%%%%%%%%%%%%%%%%%%%%%%%%%%%%%%%%%%%%%%%%%%%
%                              FI DEL DOCUMENT                                %
%%%%%%%%%%%%%%%%%%%%%%%%%%%%%%%%%%%%%%%%%%%%%%%%%%%%%%%%%%%%%%%%%%%%%%%%%%%%%%%

\end{document}
